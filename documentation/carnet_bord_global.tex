\documentclass[a4paper,12pt]{article}
\usepackage[a4paper, margin=2.4cm]{geometry}
\usepackage[utf8]{inputenc}
\usepackage{amsmath, amssymb} % Packages pour les maths
\usepackage[T1]{fontenc}
\usepackage{graphicx} 
\usepackage{caption}
\usepackage{setspace} % Espacement des lignes
\usepackage{tikz}




\title{Projet de Modélisation de la Température de la Terre}
\author{}
\date{}

\begin{document}
\maketitle
\section*{Projet}

\subsection*{14h - 15h}

\begin{itemize}
    \item Premier modèle simple : la Terre est assimilée à une \textbf{boule d'eau} (corps noir) dans le \textbf{vide}. Étude des rayonnements, première façon de modéliser l'évolution de la température la nuit.
    \item Deuxième modèle : la Terre est une \textbf{boule d'eau} dans l'\textbf{air} (atmosphère), en ignorant la convection.
    \item Création du \textbf{repo GitHub}.
    \item Création du document \LaTeX{} pour la théorie des modèles.
\end{itemize}

\subsection*{15h - 16h}

\begin{itemize}
    \item La Terre est maintenant assimilée à une \textbf{coquille}.
    \item Mise à jour du calcul de $C$ (capacité thermique).
    \item Exécution des scripts Python des modèles 1 et 2 pour constater les \textbf{changements} sur les courbes.
\end{itemize}

\subsection*{16h - 17h}

\begin{itemize}
    \item Modèle 3 : coquille, mais de l'air et non du vide à l'intérieur, donc avec une température constante et un rayonnement reçu par la croûte depuis le noyau (assemblage des modèles 1 et 2, avec température uniforme : pas de dépendance en $\theta$ ou $\phi$).
    \item Schéma explicatif du modèle 3 (Melvin).
    \item Passage des scripts pour n'afficher que de 0 à 12h en abscisse.
\end{itemize}

\subsection*{17h - 17h30}

\begin{itemize}
    \item Modèle 4 : fusion des modèles 1, 2 et 3 en prenant en compte le Soleil.
\end{itemize}

\end{document}
