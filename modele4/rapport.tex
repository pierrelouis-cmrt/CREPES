\documentclass[a4paper,12pt]{article}
\usepackage[a4paper, margin=2.4cm]{geometry}
\usepackage[utf8]{inputenc}
\usepackage{amsmath, amssymb} % Packages pour les maths
\usepackage[T1]{fontenc}
\usepackage{graphicx} 
\usepackage{caption}
\usepackage{setspace} % Espacement des lignes
\usepackage{tikz}
\usepackage{titlesec}
\titleformat*{\section}{\LARGE\bfseries}
\titleformat*{\subsection}{\Large\bfseries}
\titleformat*{\subsubsection}{\normalsize\bfseries}
\begin{document}
\section*{Modèle 4: }
\subsubsection*{Hypothèses}:
\begin{itemize}
    \item Terre modélisée avec les continents, les mers, les océans 
    \item  On néglige: 
    \begin{itemize}
        \item convection dans l'air
        \item la conduction entre centre de la terre et croûte
        \item la conduction entre croûte et air (\(P_{th,cond}\))
    \end{itemize} 
    \item  rayonnement de la croûte (\(P_{th,ray}\))
    \item On considère la température de l'atmosphère constante dans l'espace et  au cours du temps 
    \item On prend en compte l'albédo du sol en fonction de la position et du mois de l'année 
    \item la capacité thermique dépend de l'espace 
    \item On considère $T(t=0) = T_i$ \  indépendant de l'espace  
    
    
    
    
\end{itemize}

\subsubsection*{Schéma:} 
\noindent\textcolor{gray}{\rule{\linewidth}{0.4pt}}

    
\begin{center}
  \input{modele4/figures/modele 4.txt }
\end{center}
\noindent\textcolor{gray}{\rule{\linewidth}{0.4pt}}

\subsubsection*{Équations de transfert thermique}

On applique le premier principe appliqué au système \{ Un pavé de hauteur h, de surface S de la Terre \}
\ \ 

\textbf{Modélisation graphique :} 
    
    \includegraphics[width=0.8\linewidth]{}    


\textbf{Critique du modèle}
\vspace{1cm}

\end{document}