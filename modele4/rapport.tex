\documentclass[a4paper,12pt]{article}
\usepackage[a4paper, margin=2.4cm]{geometry}
\usepackage[utf8]{inputenc}
\usepackage{amsmath, amssymb} % Packages pour les maths
\usepackage[T1]{fontenc}
\usepackage{graphicx} 
\usepackage{caption}
\usepackage{siunitx}
\usepackage{setspace} % Espacement des lignes
\usepackage{tikz}
\usetikzlibrary{patterns}
\usepackage{hyperref}
\usepackage{titlesec}
\titleformat*{\section}{\LARGE\bfseries}
\titleformat*{\subsection}{\Large\bfseries}
\titleformat*{\subsubsection}{\normalsize\bfseries}
\begin{document}
\section*{Modèle 4: }
\subsubsection*{Hypothèses:}
\begin{itemize}
    \item Terre modélisée avec les continents, les mers, les océans 
    \item  On néglige: 
    \begin{itemize}
        \item convection dans l'air
        \item la conduction entre centre de la terre et croûte
        \item la conduction entre croûte et air (\(P_{th,cond}\))
    \end{itemize} 
    \item  rayonnement de la croûte (\(P_{th,ray}\))
    \item On considère la température de l'atmosphère constante dans l'espace et  au cours du temps 
    \item On prend en compte l'albédo du sol en fonction de la position et du mois de l'année 
    \item On prend en compte l'albedo des nuages.
    \item la capacité thermique dépend de la localisation
    \item On considère $T(t=0) = T_i$ \  indépendant de l'espace  
    \item On prend en compte la chaleur latente considérée avec des constantes propres à chaque continent\\
    
    
    
    
\end{itemize}

\subsubsection*{Schéma:} 
\noindent\textcolor{gray}{\rule{\linewidth}{0.4pt}}

    
\begin{center}
  \input{modele4/figures/Schéma modèle 4.txt }
\end{center}
\noindent\textcolor{gray}{\rule{\linewidth}{0.4pt}}
\vspace{0.2cm}
\subsubsection*{Équations de transfert thermique}

On applique le premier principe appliqué au système \{ Un pavé de hauteur h, de surface S de la Terre \}
\ \ 
\[
c_s \frac{dT}{dt} =(1-A_1)(1-A_2)\varphi_s-\sigma T^4+\sigma T_{atm}^4-\varphi_{lat}
\]
Avec
\begin{itemize}
    \item \(c_s\) la capacité thermique surfacique en \(J\cdot K^{-1}\cdot m^{-2}\)
     \item  \(T\) la température du système en \(K\)
    \item \(A_1\) l'albédo des nuages (sans unité)
    \item \(A_2\) l'albédo du sol (sans unité)
    \item \(\varphi_s\) le flux solaire incident en \(W \cdot m^2\)
 \item \(\sigma\) la constante de Stefan-Boltzmann \(W \cdot m^{-2} \cdot K^{-4}\)
    \item \(T_\text{atm}\) la température de l'atmosphère en \(K\)    
    
    \item \(\varphi_{lat}\) le flux de chaleur latente émis par notre système en \(W \cdot m^2\)
\end{itemize}



\vspace{0.5cm}
\subsubsection{Tabulation de l'albédo de surface en fonction de la position }
Les données ont été récupérées sur les dossiers CSV du projet (groupe B)  de l'année précédente. Il s'agit de moyennes mensuelles tabulées pour chaque position sur Terre (latitude longitude)
\subsubsection{Tabulation de l'albédo des nuages}
Obtenu par différence de l'albédo de la terre lors de ciel clair et ciel nuageux.
Moyennes mensuelles tabulées sur 1 an (1 janvier 2024 à 1 janvier 2025)
Source : CERES_EBAF-TOA_Ed4.2.1
lien du code : 

\subsubsection{Calculs capacités calorifiques en fonction de la composition des surfaces}
Données récoltées sur : 
\url{https://data.catds.fr/cpdc/Land_products/GRIDDED/L4SM/OPER/}

\begin{itemize}
    \item Moyenne faite sur une période de 1 an grâce à 12 données récoltées le matin et 12 données récoltées le soir 
    \item Les 12 données ont été prise à des périodes différentes sur l'année 
    \item La masse volumique de la Terre est approximée constante 
\end{itemize}
\[
w = \frac{\rho_w \theta}{\rho_b (1 - \theta) + \rho_w \theta}
\]
Avec \(\theta\): le taux d'humidité en \(m^3\) d'eau par \(m^3\) de sol 
\[
c_p = c_{p,\text{sec}} + w (c_{p,\text{eau}} - c_{p,\text{sec}})
\]
Avec \(w\): la fraction massique d'eau 
\begin{table}[h!]
\centering
\begin{tabular}{ll}
\textbf{Constante} & \textbf{Valeur} \\
\hline
$c_{p,\text{sec}}$ & \SI{0.80}{\kilo\joule\per\kilogram\per\kelvin} \\
$c_{p,\text{eau}}$ & \SI{4.187}{\kilo\joule\per\kilogram\per\kelvin} \\
$\rho_b$ (sol) & \SI{1300}{\kilogram\per\cubic\metre} \\
$\rho_w$ (eau) & \SI{1000}{\kilogram\per\cubic\metre} \\
\end{tabular}
\caption*{Constantes utilisées pour le calcul de $c_p$}
\vspace{1cm}
\end{table}

Détail du calcul de \( c_s \) :\\
\begin{align*}
c_p = \frac{C}{m} = \frac{C}{\rho V} = \frac{C}{\rho S h} = \frac{c_s}{\rho h}
\end{align*}
\hspace*{2cm}Avec \( C \) la capacité thermique



\subsubsection*{Modélisation graphique (à Paris) :} 
\begin{itemize}
    \item Modèle adaptatif
\end{itemize}

\begin{center}
  \includegraphics[width=0.8\linewidth]{modele4/figures/adaptatif_année.jpeg}
  \end{center}
\begin{center}
  \includegraphics[width=0.8\linewidth]{modele4/figures/adaptatif_jour.jpeg}
  \end{center}
\begin{itemize}
    \item Modèle explicite
\end{itemize}


\begin{center}
  \includegraphics[width=0.8\linewidth]{modele4/figures/explicite_année.jpeg}
  \end{center}
\begin{center}
  \includegraphics[width=0.8\linewidth]{modele4/figures/explicité_jour.jpeg}
  \end{center}
\begin{itemize}
    \item Modèle réccurent
\end{itemize}

\begin{center}
  \includegraphics[width=0.8\linewidth]{modele4/figures/récurrent_année.jpeg}
  \end{center}
\begin{center}
  \includegraphics[width=0.8\linewidth]{modele4/figures/récurent_jour.jpeg}
  \end{center}
\vspace{1cm}


\textbf{Critique du modèle}
\begin{itemize}
    \item On néglige la conducto-convexion
    \item On considère la température de l'atmosphère homogène et constante au cours du temps
    \item Les valeurs de chaleur latente sont prises constantes, ce qui n'est pas le cas en réalité
\end{itemize}


\subsection*{Chaleur latente}

\subsubsection*{C quoi ?}

Le flux de chaleur latente correspond à la puissance surfacique positive(reçue) ou négative(fournie) afin d'évaporer ou de condenser l'eau de la surface de la Terre.
On découpe la Terre de plusieurs manières, par mers et océans, et par continents. 
On parle d'une moyenne faite par continents :
\begin{itemize}
    \item Les mers et océans : 108.8 $W/m^2$
    \item L'Asie : 28.8 $W/m^2$
    \item L'Afrique : 45.1 $W/m^2$
    \item L'Europe : 38.1 $W/m^2$
    \item L'Amérique du Nord : 36.5 $W/m^2$
    \item L'Amérique du Sud : 73.1 $W/m^2$
    \item L'Océanie : 31.9 $W/m^2$
\end{itemize}

On suppose que la pression atmosphérique de la Terre est constante à la surface. On applique le premier principe thermodynamique:

\vspace{1cm}
\begin{equation}
    dH = \delta Q
\end{equation}
\begin{equation}
    dm \Delta h_{vap} = P \times dt
\end{equation}
\begin{equation}
    Dm \Delta h{vap} = P

\end{equation}



\end{document}